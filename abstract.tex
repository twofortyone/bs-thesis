%\setlength{\topmargin}{-.5in}

% \epigraph{\textit{It is not a dream. It is a simple feat of scientific electrical engineering. 
% Electric power can drive the world's machinery without the need of coal, oil or gas.
% Although perhaps humanity is not yet sufficiently advanced to be willingly lead by the 
% inventor's keen searching sense. Perhaps it is better in this present world of ours where 
% a revolutionary idea may be hampered in its adolescence. All this that was great in the past 
% was ridiculed, condemned, combatted, suppressed only to emerge all the more triumphantly from 
% the struggle. [...] Our duty is to lay the foundation for those who are to come and to point 
% the way, yes humanity will advance with giant strides. We are whirling through endless space
% with an inconceivable speed, all around everything is spinning, everything is moving,
% everywhere there is energy.}}{--- Nicola Tesla}

\chapter*{Abstract}
\addcontentsline{toc}{chapter}{Abstract}

% give a background
% Summarized the problem 
% State the solution 
% the main aim is to present a methodology for Service Restoration using Reinforcement Learning 
% The proposed Machine Learning technique for solving this problem is Reinforcement Learning 
% Explain the methodology: how is works and what are the constraints preserved 
% How this was tested 

Distribution Networks (DNs) are highly susceptible to faults, which affects their quality and reliability. This thesis proposes a novel Service Restoration approach to help DNs to automatically and intelligently resupply the out-of-service un-faulted customers after a fault occurs. The presented approach is developed with the use of Reinforcement Learning techniques and it is integrated with OpenDSS. 

\vspace*{5ex}
\textbf{\textit{Keywords:}} DA, Reinforcement Learning, Q-Learning, Service Restoration, OpenDSS.

