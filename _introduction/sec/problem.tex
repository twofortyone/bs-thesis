\section{Problem Statement and Research Objectives}
\label{ch-introduction:sec:problem}

Service Restoration uses network reconfiguration methods to change the DN topology and resupply the out-of-service 
un-faulted customers \cite{Gholami2015}\cite{Shen2018}. This reconfiguration is carried out through 
switching operations, considering typical DNs have normally closed (NC) sectionalizing switches and normally 
opened (NO) tie switches \cite{Zidan2017} \cite{Sanches2014}. I.e., the main goal of SR is to find 
optimal switching sequences for the DN \cite{Latare2017}.

This procedure must maximize the number of loads restored, minimize the number of switching operations, and 
maintain operational and topological constraints \cite{Abu-Elanien2018} \cite{Gholami2015}. 
Nevertheless, and as noted in the formulation problem (Section \ref{ch-literature:sec:overview}), SR is computationally demanding because it is multi-objective, 
non-linear, and operational constrained. And also, it requires to explore a large number of switching operation 
combinations \cite{Shen2018} \cite{Sanches2014}. 

SR algorithm uses one or many techniques to obtain a solution, among which are expert systems, heuristic 
algorithms, meta-heuristic algorithms, graph theory, multi-agent systems, and mathematical programming \cite{Shen2018}. 
Furthermore, they can be implemented under centralized or decentralized approaches \cite{Zidan2017}. The control approaches 
and their corresponding techniques are reviewed, in further detail, in section \ref{ch-literature:sec:overview} and 
summarized in table \ref{ch-introduction:tab:sr_techniques}. 

\begin{table}
    \centering
    \caption{SR techniques comparison \cite{Abu-Elanien2018} \cite{Zidan2017}}
    \label{ch-introduction:tab:sr_techniques}
    \begin{tabular}{|l|c|c|c|}
        \hline
        \multicolumn{1}{|c|}{\textbf{Method}}                                        & \multicolumn{1}{c|}{\textbf{\begin{tabular}[c]{@{}c@{}}Optimal \\ solution\end{tabular}}} & \multicolumn{1}{c|}{\textbf{\begin{tabular}[c]{@{}c@{}}System \\ dependency\end{tabular}}} & \multicolumn{1}{c|}{\textbf{\begin{tabular}[c]{@{}c@{}}Self-learning \\ capacity\end{tabular}}} \\ \hline
        \textbf{Expert systems}                                                      & No                                                                                        & Yes                                                                                       & No                                                                                              \\ \hline
        \textbf{Heuristics}                                                          & No                                                                                        & Yes                                                                                       & No                                                                                              \\ \hline
        \textbf{Meta-heuristics}                                                     & No                                                                                        & Yes                                                                                       & No                                                                                              \\ \hline
        \textbf{\begin{tabular}[c]{@{}l@{}}Mathematical \\ programming\end{tabular}} & Yes                                                                                       & No                                                                                        & No                                                                                              \\ \hline
        \end{tabular}
\end{table}


Table \ref{ch-introduction:tab:sr_techniques} compares the different methodologies used to develop SR 
algorithms and shows their limitations. Most methodologies cannot obtain an optimal solution, 
but improvements in processing time \cite{Abu-Elanien2018}; thus, the solution optimality is sacrificed to obtain a faster solution. 
On the other hand, these methodologies, except mathematical programming, are system dependent and do not 
have the ability to self-learn, becoming out of date when system variations occur \cite{Zidan2017}.

Therefore, these limitations imply a need for human intervention to update their rules or knowledge bases 
and make the methodologies readapt to the new changes.

\subsection{General Purpose}

According to the previous data, the focus of this work is to develop a scalable Service Restoration algorithm 
capable of self-healing Distribution Networks by automatic learning the optimal sequence of switching 
operations to lead DNs to the best possible state after a fault occurs. Moreover, 
the challenge of automatic learning behavior also includes continuous learning, hence the system is 
constantly exploring to improve the sequence and adapt to changes in the network. Last but not least, 
the optimization objectives and constraints formalized in Chapter \ref{ch-literature} are also taken into account.

This work also aims to develop a validation system of the SR algorithm, which allows simulations of a 
practical distribution network and verification of its operational and topological constraints, 
through the integration of a co-simulation engine. 

\begin{enumerate}[
    labelindent=*,
    style=multiline,
    leftmargin=*,
    label=\textbf{Objective~\arabic*}]
        %\item Review existing methods for Service Restoration. 
        %\item Design a solution that improve the existing methods. 
        \item Develop a Service Restoration algorithm for Distribution Networks capable of self-learning to obtain an optimal solution that accomplishes operational and topological constraints by implementing a Reinforcement Learning technique.
        \item Develop a validation system that emulates and measures the parameters of a Distribution Network, in order to test the proposed SR algorithm performance, by integrating a co-simulation engine.
        %\item Develop a validation system by using co-simulation engine. 
\end{enumerate}
    

