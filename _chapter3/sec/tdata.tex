\section{RL Training Results}
The Service Restoration algorithm obtains the restoration plan through reinforcement learning training, 
as achieved in Chapter \ref{ch1}. Training scenario turns the network parameters into a Markov 
Decision Process, which defines the behavior and size of the algorithm operations.
Therefore, the numbers of lines and switches of the two test cases presented in Table \ref{ch2:tab:dn_data} are 
directly related to the number of states $n(S)$ and actions $n(A)$ presented in the following tables.

Additionally, Tables \ref{ch3:tab:td33}, \ref{ch3:tab:td123_1}, and \ref{ch3:tab:td123_2} show the 
elapsed time $T[min]$, the number of failures restored $n(FR)$, and the restoration rate $RR$ for each 
training relying on switch configurations in the IEEE 33 and the IEEE 123 Node Test Feeder.

\begin{table}
    \centering
    \caption{IEEE 33 - Training parameters 2sw - 5sw}
    \label{ch3:tab:td33}
    \begin{tabular}{lcccc} 
    \hline
    \multicolumn{1}{c}{\multirow{2}{*}{ \textbf{Data} }} & \multicolumn{4}{l}{\textbf{IEEE 33 Node Test Feeder}~ ~ }      \\ 
    \cline{2-5}
    \multicolumn{1}{c}{}                                 & \textbf{2sw}  & \textbf{3sw}  & \textbf{4sw}  & \textbf{5sw}   \\ 
    \hline
     \textbf{n(S)}                                       & 128           & 256           & 512           & 1024           \\
    \textbf{n(A)}                                        & 12            & 13            & 14            & 15             \\
    \textbf{T [min]}                                       & 1.3           & 2.9           & 5.6           & 10             \\
    \textbf{n(FR)}                                       & 20            & 28            & 31            & 31             \\
    \textbf{RR}                                          & 1             & 1             & 1             & 1              \\
    \hline
    \end{tabular}
    \end{table}
\begin{table}
    \centering
    \caption{IEEE 123 - Training parameters 5sw - 10sw}
    \label{ch3:tab:td123_1}
    \begin{tabular}{lcccccc} 
    \hline
    \multicolumn{1}{c}{\multirow{2}{*}{ \textbf{Data} }} & \multicolumn{6}{c}{\textbf{IEEE 123 Node Test Feeder} }                                         \\ 
    \cline{2-7}
    \multicolumn{1}{c}{}                                 & \textbf{5sw}  & \textbf{6sw}  & \textbf{7sw}  & \textbf{8sw}  & \textbf{9sw}  & \textbf{10sw}   \\ 
    \hline
     \textbf{n(S)}                                       & 3.744         & 7.588         & 14.976        & 29.952        & 59.904        & 119.808         \\
    \textbf{n(A)}                                        & 5             & 6             & 7             & 8             & 9             & 10              \\
    \textbf{T [min]}                                       & 7.3           & 8.4           & 11.0          & 15.5          & 17.7          & 41.3            \\
    \textbf{n(FR)}                                       & 0             & 0             & 14            & 22            & 22            & 41              \\
    \textbf{RR}                                          & 0             & 0             & 1             & 1             & 1             & 1               \\
    \hline
    \end{tabular}
    \end{table}
\begin{table}
    \centering
    \caption{IEEE 123 - Training parameters 11sw - 15sw}
    \label{ch3:tab:td123_2}
    \begin{tabular}{lccccc} 
    \hline
    \multicolumn{1}{c}{\multirow{2}{*}{ \textbf{Data} }} & \multicolumn{5}{c}{\textbf{IEEE 123 Node Test Feeder}~ ~ ~ ~~ }                     \\ 
    \cline{2-6}
    \multicolumn{1}{c}{}                                 & \textbf{11sw}  & \textbf{12sw}  & \textbf{13sw}  & \textbf{14sw}  & \textbf{15sw}   \\ 
    \hline
     \textbf{n(S)}                                       & 239.616        & 479.232        & 958.464        & 1.916.928      & 3.833.856       \\
    \textbf{n(A)}                                        & 11             & 12             & 13             & 14             & 15              \\
    \textbf{T [min]}                                       & 72.5           & 145.0          & 290.3          & 602.4          & 1198.2          \\
    \textbf{n(FR)}                                       &     43           &     45           &    45            &  46              &   51              \\
    \textbf{RR}                                          & 1              & 1              & 1              & 1              & 1               \\
    \hline
    \end{tabular}
    \end{table}
