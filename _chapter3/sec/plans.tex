\section{Service Restoration Plans}

The Service Restoration Plan specifies the optimal switching operations to restore the maximum number of loads
maintaining the operational and topological constraints after a fault occurs. 
The execution of this plan is carried out in the Production scenario, that is, once the system is already trained.

Tables \ref{ch2:tab:33_actions} and \ref{ch2:tab:123_actions} present the restoration plans for IEEE 33 Node Test Feeder with 5 switches 
and IEEE 123 Node Test Feeder with 10 switches, respectively. Additionally, they contain the number of actions required $n(A)$, 
the processing time (T[s]), the number of isolated loads before $n(IL)_{pre}$ and after $n(IL)_{post}$  restoration, and the restoration rate ($RR$).
The latter is calculated as follows: $ RR = \frac{n(IL)_{prev}-n(IL)_{post}}{n(IL)_{prev}}$.

The Service Restoration Plans for the other switch configurations of the test cases are presented in Appendix \ref{appx-a:additional-results}.
\begin{table}
    \centering
    \caption{SR plan - IEEE 33 Node Test Feeder - 5 switches}
    \label{ch2:tab:33_actions}
    \begin{tabular}{llccccc}
    \hline
    \textbf{\begin{tabular}[c]{@{}l@{}}Faulted\\  line\end{tabular}} & \textbf{Actions}    & \textbf{N(A)} & \textbf{T {[}s{]}} & \textbf{N(IL)$_{pre}$} & \textbf{N(IL)$_{post}$} & \textbf{RR} \\ \hline
    \textit{\textbf{l1}}                                             & {[}('s33', None){]} & 0             & 0.005               & 32             & 32             & 0           \\
    \textit{\textbf{l2}}                                             & {[}('s35', 1){]}    & 1             & 0.005               & 26             & 0              & 1           \\
    \textit{\textbf{l3}}                                             & {[}('s37', 1){]}    & 1             & 0.005              & 22             & 0              & 1           \\
    \textit{\textbf{l4}}                                             & {[}('s37', 1){]}    & 1             & 0.005               & 21             & 0              & 1           \\
    \textit{\textbf{l5}}                                             & {[}('s37', 1){]}    & 1             & 0.005               & 20             & 0              & 1           \\
    \textit{\textbf{l6}}                                             & {[}('s36', 1){]}    & 1             & 0.005               & 12             & 0              & 1           \\
    \textit{\textbf{l7}}                                             & {[}('s36', 1){]}    & 1             & 0.004               & 11             & 0              & 1           \\
    \textit{\textbf{l8}}                                             & {[}('s36', 1){]}    & 1             & 0.005               & 10             & 0              & 1           \\
    \textit{\textbf{l9}}                                             & {[}('s36', 1){]}    & 1             & 0.004               & 9              & 0              & 1           \\
    \textit{\textbf{l10}}                                            & {[}('s36', 1){]}    & 1             & 0.004               & 8              & 0              & 1           \\
    \textit{\textbf{l11}}                                            & {[}('s36', 1){]}    & 1             & 0.005               & 7              & 0              & 1           \\
    \textit{\textbf{l12}}                                            & {[}('s36', 1){]}    & 1             & 0.005               & 6              & 0              & 1           \\
    \textit{\textbf{l13}}                                            & {[}('s36', 1){]}    & 1             & 0.004               & 5              & 0              & 1           \\
    \textit{\textbf{l14}}                                            & {[}('s36', 1){]}    & 1             & 0.005               & 4              & 0              & 1           \\
    \textit{\textbf{l15}}                                            & {[}('s36', 1){]}    & 1             & 0.005               & 3              & 0              & 1           \\
    \textit{\textbf{l16}}                                            & {[}('s36', 1){]}    & 1             & 0.004               & 2              & 0              & 1           \\
    \textit{\textbf{l17}}                                            & {[}('s36', 1){]}    & 1             & 0.004               & 1              & 0              & 1           \\
    \textit{\textbf{l18}}                                            & {[}('s35', 1){]}    & 1             & 0.005               & 5              & 0              & 1           \\
    \textit{\textbf{l19}}                                            & {[}('s35', 1){]}    & 1             & 0.005               & 4              & 0              & 1           \\
    \textit{\textbf{l20}}                                            & {[}('s35', 1){]}    & 1             & 0.005               & 2              & 0              & 1           \\
    \textit{\textbf{l21}}                                            & {[}('s35', 1){]}    & 1             & 0.005               & 1              & 0              & 1           \\
    \textit{\textbf{l22}}                                            & {[}('s37', 1){]}    & 1             & 0.005               & 3              & 0              & 1           \\
    \textit{\textbf{l23}}                                            & {[}('s37', 1){]}    & 1             & 0.005               & 2              & 0              & 1           \\
    \textit{\textbf{l24}}                                            & {[}('s37', 1){]}    & 1             & 0.005               & 1              & 0              & 1           \\
    \textit{\textbf{l25}}                                            & {[}('s37', 1){]}    & 1             & 0.004               & 7              & 0              & 1           \\
    \textit{\textbf{l26}}                                            & {[}('s37', 1){]}    & 1             & 0.005               & 6              & 0              & 1           \\
    \textit{\textbf{l27}}                                            & {[}('s37', 1){]}    & 1             & 0.005               & 5              & 0              & 1           \\
    \textit{\textbf{l28}}                                            & {[}('s37', 1){]}    & 1             & 0.005               & 4              & 0              & 1           \\
    \textit{\textbf{l29}}                                            & {[}('s36', 1){]}    & 1             & 0.005               & 3              & 0              & 1           \\
    \textit{\textbf{l30}}                                            & {[}('s36', 1){]}    & 1             & 0.005               & 3              & 0              & 1           \\
    \textit{\textbf{l31}}                                            & {[}('s36', 1){]}    & 1             & 0.005               & 2              & 0              & 1           \\
    \textit{\textbf{l32}}                                            & {[}('s36', 1){]}    & 1             & 0.004               & 1              & 0              & 1           \\ \hline
    \end{tabular}
    \end{table}
\begin{table}
    \centering
    \caption{SR plan - IEEE 123 Node Test Feeder - 10 switches }
    \label{ch2:tab:123_actions}
    \begin{tabular}{llccccc}
    \hline
    \textbf{\begin{tabular}[c]{@{}l@{}}Faulted\\  line\end{tabular}} & \textbf{Actions}             & \textbf{N(A)} & \textbf{T {[}s{]}} & \textbf{N(IL)$_{pre}$} & \textbf{N(IL)$_{post}$} & \textbf{RR} \\ \hline
    \textit{\textbf{l116}}                                           & {[}('sw7', 1){]}             & 1             & 0.016              & 52             & 0              & 1           \\
    \textit{\textbf{l52}}                                            & {[}('sw7', 1){]}             & 1             & 0.016              & 51             & 0              & 1           \\
    \textit{\textbf{l53}}                                            & {[}('sw7', 1){]}             & 1             & 0.017              & 50             & 0              & 1           \\
    \textit{\textbf{l55}}                                            & {[}('sw7', 1){]}             & 1             & 0.016              & 48             & 0              & 1           \\
    \textit{\textbf{l58}}                                            & {[}('sw7', 1){]}             & 1             & 0.018              & 46             & 0              & 1           \\
    \textit{\textbf{l117}}                                           & {[}('sw7', 1){]}             & 1             & 0.016              & 38             & 0              & 1           \\
    \textit{\textbf{l67}}                                            & {[}('sw7', 1){]}             & 1             & 0.017              & 21             & 0              & 1           \\
    \textit{\textbf{l73}}                                            & {[}('sw7', 1){]}             & 1             & 0.016              & 18             & 0              & 1           \\
    \textit{\textbf{l77}}                                            & {[}('sw7', 1){]}             & 1             & 0.017              & 8              & 0              & 1           \\
    \textit{\textbf{l86}}                                            & {[}('sw7', 1){]}             & 1             & 0.016              & 7              & 0              & 1           \\
    \textit{\textbf{l88}}                                            & {[}('sw7', 1){]}             & 1             & 0.017              & 5              & 0              & 1           \\
    \textit{\textbf{l90}}                                            & {[}('sw7', 1){]}             & 1             & 0.016              & 4              & 0              & 1           \\
    \textit{\textbf{l92}}                                            & {[}('sw7', 1){]}             & 1             & 0.017              & 3              & 0              & 1           \\
    \textit{\textbf{l94}}                                            & {[}('sw7', 1){]}             & 1             & 0.016              & 2              & 0              & 1           \\
    \textit{\textbf{l17}}                                            & {[}('sw7', 1){]}             & 1             & 0.017              & 1              & 0              & 1           \\
    \textit{\textbf{l34}}                                            & {[}('sw7', 1){]}             & 1             & 0.018              & 2              & 0              & 1           \\
    \textit{\textbf{l68}}                                            & {[}('sw10', 1){]}            & 1             & 0.016              & 13             & 0              & 1           \\
    \textit{\textbf{l118}}                                           & {[}('sw10', 1){]}            & 1             & 0.016              & 10             & 0              & 1           \\
    \textit{\textbf{l101}}                                           & {[}('sw10', 1){]}            & 1             & 0.016              & 7              & 0              & 1           \\
    \textit{\textbf{l105}}                                           & {[}('sw10', 1){]}            & 1             & 0.016              & 5              & 0              & 1           \\
    \textit{\textbf{l63}}                                            & {[}('sw10', 1){]}            & 1             & 0.017              & 5              & 0              & 1           \\
    \textit{\textbf{l62}}                                            & {[}('sw10', 1){]}            & 1             & 0.016              & 6              & 0              & 1           \\
    \textit{\textbf{l61}}                                            & {[}('sw10', 1){]}            & 1             & 0.017              & 7              & 0              & 1           \\
    \textit{\textbf{l114}}                                           & {[}('sw9', 1){]}             & 1             & 0.017              & 16             & 0              & 1           \\
    \textit{\textbf{l36}}                                            & {[}('sw9', 1){]}             & 1             & 0.016              & 12             & 0              & 1           \\
    \textit{\textbf{l41}}                                            & {[}('sw9', 1){]}             & 1             & 0.016              & 11             & 0              & 1           \\
    \textit{\textbf{l43}}                                            & {[}('sw9', 1){]}             & 1             & 0.018              & 9              & 0              & 1           \\
    \textit{\textbf{l45}}                                            & {[}('sw9', 1){]}             & 1             & 0.017              & 7              & 0              & 1           \\
    \textit{\textbf{l48}}                                            & {[}('sw9', 1){]}             & 1             & 0.016              & 5              & 0              & 1           \\
    \textit{\textbf{l49}}                                            & {[}('sw9', 1){]}             & 1             & 0.020              & 2              & 0              & 1           \\
    \textit{\textbf{l50}}                                            & {[}('sw9', 1){]}             & 1             & 0.017              & 1              & 0              & 1           \\
    \textit{\textbf{l31}}                                            & {[}('sw9', 1){]}             & 1             & 0.017              & 1              & 0              & 1           \\
    \textit{\textbf{l30}}                                            & {[}('sw9', 1){]}             & 1             & 0.017              & 2              & 0              & 1           \\
    \textit{\textbf{l26}}                                            & {[}('sw9', 1){]}             & 1             & 0.017              & 3              & 0              & 1           \\
    \textit{\textbf{l24}}                                            & {[}('sw9', 1){]}             & 1             & 0.017              & 6              & 0              & 1           \\
    \textit{\textbf{l22}}                                            & {[}('sw9', 1), ('sw6', 0){]} & 2             & 0.018              & 7              & 0              & 1           \\
    \textit{\textbf{l19}}                                            & {[}('sw9', 1){]}             & 1             & 0.017              & 8              & 0              & 1           \\
    \textit{\textbf{l12}}                                            & {[}('sw7', 1){]}             & 1             & 0.018              & 3              & 0              & 1           \\
    \textit{\textbf{l115}}                                           & {[}('sw5', None){]}          & 0             & 0.017              & 91             & 91             & 0           \\
    \textit{\textbf{l5}}                                             & {[}('sw7', None){]}          & 0             & 0.016              & 2              & 2              & 0           \\
    \textit{\textbf{l1}}                                             & {[}('sw3', None){]}          & 0             & 0.016              & 1              & 1              & 0           \\ \hline
    \end{tabular}
    \end{table}