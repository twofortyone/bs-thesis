\section{Service Restoration}
\label{ch-literature:sec:sr}
Service Restoration can be performed by centralized and decentralized approaches \cite{Zidan2017}\cite{Yip2017}.
The first one includes expert systems, heuristics, metaheuristics, and mathematical programming methodologies; 
the second relies on Multiagent Systems (MAS) \cite{Chellaswamy2019}. 

The two types of approaches, and their corresponding methodologies, are reviewed in section \ref{ch-literature:sec-sr:approaches}. 
However, both approaches tend to solve the Service Restoration problem that can be 
formalized as a general optimization problem, as suggested in section \ref{ch-literature:sec-sr:problem_formulation}. 


\subsection{General SR Problem Formulation}
\label{ch-literature:sec-sr:problem_formulation}
Service Restoration can be formalized as an optimization problem \cite{Huang2016}. 

The main objective function aims to maximize the number of loads restored (equation 
\ref{ch-introduction:equ:max_loads}), while the second aims to minimize the number of operations performed 
to resupply the aforementioned loads (equation \ref{ch-introduction:equ:min_ops}) \cite{Shen2018}. 

\begin{equation} 
\text{max} \sum_{i=1}^{n} L_i \cdot k_i
	\label{ch-introduction:equ:max_loads}
\end{equation}
where $L_i$ is the load at bus \textit{i}, and $k_i$ is its status (1: restored; 0: not restored).
\begin{equation} 
    \text{min } n_{ops} 
    \label{ch-introduction:equ:min_ops}
    \end{equation}
where $n_{ops}$ is the number of switching operations. 

\subsubsection*{Constraints}
The presented objective functions are subject to operational and topological constraints \cite{Abu-Elanien2018}. 

\begin{enumerate}
    \item \textbf{Radial topology:} Radial network topology should be preserved.
    \item \textbf{Bus voltage limits:} Nodal voltage $V_i$ should be within the minimum $V_{min}$ and maximum $V_{max}$ voltage limit: $V_{min} \leq V_i \leq V_{max}$ 
    \item \textbf{Line current limits:} Line current $I_i$ should be under the maximum acceptable line current $ I_{max}$: $I_i \leq I_{max}$.
\end{enumerate}

\subsection{Control approaches and methodologies}
\label{ch-literature:sec-sr:approaches}
\subsubsection*{Centralized approach}
The centralized approach is the principal control approach used in Service Restoration \cite{Abu-Elanien2018}. 
It captures and processes measurements and status information of the entire distribution network to obtain 
an optimal 
SR solution \cite{Koch-Ciobotaru2014}, because of that it requires an extensive computational capacity and a low latency system 
communication \cite{Zidan2017} \cite{Shen2018}. 
Nonetheless, the communication system can be a single point of failure \cite{Shen2018}. 

Among its methodologies are expert system, heuristics, metaheuristic, and mathematical programming 
\cite{Chellaswamy2019}. 



\begin{enumerate}
    \item \textbf{Expert system:} creates rules relying on expert knowledge \cite{Shen2018}. 
    It is a successful way to SR, however, its solution depends on the system and does not guarantee 
    optimality. Besides, its maintenance is expensive for large networks \cite{Zidan2017} \cite{Shen2018}. 
    \item \textbf{Heuristics:} partition and approximate the SR problem by 
    using expert knowledge to obtain a feasible solution. Heuristics present an improvement in time 
    processing, but a difficulty to obtain an optimal solution \cite{Abu-Elanien2018}\cite{Shen2018}. 
    \item \textbf{Mathematical programming:} is a direct method to address the SR optimization problem 
    by obtaining an optimal solution. Although, it demands substantial processing time for large networks
    \cite{Abu-Elanien2018}. 
    \item \textbf{Metaheuristics:} solves the SR problem as an optimization problem, as well as Mathematical Programming (MP),  
    but with less processing time than MP and with a better solution than Heuristics \cite{Abu-Elanien2018}. 
    This is performed by simplifying a problem with many objectives into a problem with a single equivalent objective \cite{Shen2018}. 
     However, the processing time is still huge for practical networks \cite{Zidan2017}. 
\end{enumerate}

\subsubsection*{Decentralized approach}
Decentralized approach distributes the intelligence across the distribution network \cite{Shen2018}, as opposed 
to unique control intelligence in the centralized approach. This structure requires a more robust 
and reliable communication system, but it overcomes the centralized approach disadvantages of huge 
computational capacity and one single point of failure \cite{Zidan2017} \cite{Abu-Elanien2018}. However, since in this approach, the 
agent does not know all the distribution network but its neighbor network, the solution 
obtained is suboptimal \cite{Koch-Ciobotaru2014}. 
This approach is implemented through Multiagent Systems (MAS). 


Figure \ref{ch-literature:tab:control_appraches} summarizes the strengths and limitations of each of the approaches discussed above.
\begin{table}
\centering
\caption{Service Restoration Control approaches}
\label{ch-literature:tab:control_appraches}
\begin{tabular}{|l|l|l|}
\hline
\textbf{Approach}      & \textbf{Strengths}                                                       & \textbf{Limitations}                                                                       \\ \hline
\textbf{Centralized}   & \begin{tabular}[c]{@{}l@{}}- Optimal solution \\ \end{tabular}          & \begin{tabular}[c]{@{}l@{}}- Huge computing capacity\\ - Single point of failure\end{tabular} \\ \hline
\textbf{Decentralized} & \begin{tabular}[c]{@{}l@{}}- Fast solution \\ - Less data\end{tabular} & \begin{tabular}[c]{@{}l@{}}- Suboptimal solution\\ \end{tabular}                          \\ \hline
\end{tabular}
\end{table}
%\begin{enumerate}
%    \item \textbf{Multiagent systems:}
%\end{enumerate}

%centralized

%computes the global optimal
%reconfiguration of the grid, considering all the constraints and using measurements 
%and status information for all across the grid.
%\cite{Koch-Ciobotaru2014}

%requires knowledge of the entire structure of the system
%\cite{Koch-Ciobotaru2014}

%the main control approach for reconfiguration problem
%Most of the existing studies adopted a centralized control approach
%\cite{Abu-Elanien2018}


%low latency communication system to 
%requires/transfer large/huge amounts of data among field agents and control centers (low latency)
%\cite{Zidan2017} \cite{Chellaswamy2019} \cite{Shen2018}

%thus requiring enormous/expensive computational capacity
%\cite{Chellaswamy2019} \cite{Shen2018}

%centralized methods suffer from a single point failure risk
%\cite{Zidan2017} \cite{Shen2018}

%Heuristic, metaheuristic, expert system, and mathematical solutions
%\cite{Chellaswamy2019}

% -------------------------
%expert systems 
%transfers the expert knowledge into some rules

%the maintenance of a large-scale ES is very costly and the optimality of the solution cannot be guaranteed
%\cite{Shen2018}

%The expert system approach is considered a successful way to solve SR problems; however, maintenance of large-scale expert systems has turned out to be costly. Moreover, expert system rules are system-specific, and changed with system variations
%\cite{Zidan2017}

%heuristics
%While heuristic algorithms can obtain a feasible solution
%quickly, they still require specialists’ knowledge and it is hard to derive an optimal solution.
%\cite{Shen2018}

%approaches require extensive computational time when applied for practical distribution systems
%\cite{Zidan2017}

%The meta-heuristic algorithms take less time than mathe-
%matical programming to solve the problem especially in large systems. However, the computational 
%time still considerable in case of large systems. Moreover, the data structure may affect the
%computational time of the meta-heuristic algo- rithms. Poorly prepared data structure can increase 
%the com- putational time dramatically. Although the solution of meta-heuristic algorithms is near 
%optimal solution but it is not optimal. Generally, the solution is better than the heuristic solution 
%and is acceptable by many operation engineers.
%\cite{Abu-Elanien2018}

%Meta- heuristic solve the problem based on the mathematical optimization model
%In conventional meta-heuristic algorithms , a multi- objective problem is converted into an
%equivalent single objective problem by weighting factors, and the admissible switching sequence 
%is not given
%\cite{Shen2018}

%\textbf{Descentralized}

%Decentralized approaches are based on direct peer-to-peer communication between monitoring devices
%\cite{Zidan2017}

%the multi- agent system (MAS) is developed to distribute the intelligence and control actions
% to the component level
%\cite{Shen2018}

%decentralized approach has the advantage of computation speed as it computes a local optimal s
%olution for the grid reconfiguration
%\cite{Koch-Ciobotaru2014}

%Multi-agent systems are systems with multiple interacting elements (agents). Agents are computer 
%systems that are cap- able of deciding what to do in order to satisfy the assigned job to them, 
%and they are also capable of interacting with other agents to perform the overall system goals .

%the necessity of effi- cient and reliable communication system may limit the application 
%of decentralized approaches on full size distribu- tion systems
%suboptimal plan, and this is one of the drawbacks of the decentralized approaches
%\cite{Abu-Elanien2018}
