\section{Reinforcement learning}
\label{ch-literature:sec:rl}

Accordingly to the main goal of this work, there is a need to provide the ability of autonomous 
learning to the Service Restoration algorithm, by which it can find the optimal sequence of 
switching operations independently of system variations and without human interventions.
For this reason, Reinforcement Learning plays an important role in  the development of the SR algorithm. 


Reinforcement Learning is a machine learning technique between supervised and unsupervised learning. 
It consists of an interaction between an agent and an environment through which the agent learns to achieve 
RL goals, previously defined. This interaction is two-way communication: first, the environment sends its 
current state to the agent; then, the agent chooses an action, based on that state, and returns its decision to the environment. 

Consequently, actions attempt to modify the environment (i.e., an agent selects the action, and the environment executes it) 
which changes the previous state and leads to a new one.  
The new state fixes new environment parameters that the environment itself must reward according to 
learning objectives. Additionally, these parameters allow the agent to identify a terminal state. 

As a result, the agent gets an immediate reward on the last action-state 
pair and updates its value based on its value function. It is to be noted that the agent's goal is to 
maximize the reward, and the value function depends on the selected RL algorithm, among which are Dynamic Programming, 
Monte Carlo, Q-learning, Sarsa, and Dyna Q.

The theory presented in this section is based on the work conducted by Sutton and Barto \cite{Sutton2014}. 

\subsection{Q-learning}
Q learning is one of the Temporal Difference algorithms. It learns directly from experience and 
updates its value function after each interaction, due to it combines the best features 
of both Monte Carlo and Dynamic Programming methods. 

This tabular RL algorithm saves the action-state value estimates in an $m$ by $n$ matrix, where $m$ 
represents the number of states and $n$ the number of actions, so each position corresponds to an 
action-state value.  After each interaction, the agent updates the corresponding action-state 
value by using Equation \ref{ch-literature:equ:q_value} \cite{Sutton2014}. 
\begin{equation}
Q(S_t, A_t) = Q(S_t, A_t) + \alpha \cdot [R_{t+1} + \gamma \cdot max_a Q(S{t+1,a}) - Q(S_t, A_t)]
\label{ch-literature:equ:q_value}
\end{equation}
