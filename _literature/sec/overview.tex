\section{Overview}
\label{ch-literature:sec:overview}

Many authors have referred and researched the importance of Distribution Automation (DA), and its 
advantages, to improve the quality and reliability of Distribution Networks \cite{Zidan2017} 
\cite{Yao2018} \cite{Abu-Elanien2018} \cite{Madani2015}. 

One of the goals of DA is to obtain a self-healing network capable of 
automatically removing temporary faults throughout the restoration of the un-faulted areas, 
which provides benefits to both utilities and customers. Utilities will improve their reliability 
indices and their profits because they reduce penalties from regulators and costs of restoration. 
In addition, customers will obtain a more reliable and continuous power supply \cite{Angelo2013}. 

This self-healing network can be achieved with SR methodologies 
\cite{USDepartmentofEnergy2016} \cite{Yokoyama2012} \cite{Koch-Ciobotaru2014} \cite{Zidan2012}.
